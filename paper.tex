\documentclass[11pt]{article}

\usepackage[usenames,dvipsnames,svgnames,table]{xcolor}
\usepackage{amsmath, amsthm, stmaryrd, scalerel}
\usepackage{mathtools}
\usepackage{graphicx}
\usepackage{fontspec}
\usepackage{geometry}
\usepackage{proof}
\usepackage{supertabular}
\usepackage{unicode-math}
\usepackage{xltxtra}
\usepackage{xunicode}

\usepackage{setspace,url,hyperref}

\definecolor{dark-gray}{gray}{0.20}

% fonts
% \setromanfont[Mapping=tex-text, Ligatures=Rare, Numbers=OldStyle]{Linux Libertine O}
\setmathfont{Asana Math}

\newtheorem{thm}{Theorem}[section]
\newtheorem{prop}[thm]{Proposition}
\newtheorem{lem}[thm]{Lemma}
\newtheorem{cor}[thm]{Corollary}
\theoremstyle{definition}
\newtheorem{definition}[thm]{Definition}
\newtheorem{example}[thm]{Example}
\theoremstyle{remark}
\newtheorem{remark}[thm]{Remark}
\numberwithin{equation}{section}

\def\IModeColorName{MidnightBlue}
\def\OModeColorName{Maroon}
\newcommand\IMode[1]{{\color{\IModeColorName}{#1}}}
\newcommand\OMode[1]{{\color{\OModeColorName}{#1}}}
\newcommand\Quote[1]{{\color{Gray}{«}}#1{\color{Gray}{»}}}

\newcommand\HypJ[2]{#1\ \ (#2)}
\newcommand\GenJ[2]{\vert_{\IMode{#1}}\; #2}
\newcommand\JJ{J}
\newcommand\RSet{\mathfrak{R}}
\newcommand\JSet{\mathfrak{J}}

\newcommand\IsTac[1]{\IMode{#1}\ \textit{tactic}}

\newcommand\MkSet[1]{\left\{#1\right\}}
\newcommand\MkFam[3]{\MkSet{#1_{#2}}_{#2\in #3}}
\newcommand\Member[2]{\IMode{#1}\in\IMode{#2}}

\newcommand\MkValence[3]{#1;#2\,.\, #3}
\newcommand\MkBTm[3]{#1;#2\,.\, #3}
\newcommand\MkArity[2]{\left(#1\right) #2}
\newcommand\IsArity[1]{\IMode{#1}\ \textit{arity}}
\newcommand\IsValence[1]{\IMode{#1}\ \textit{valence}}
\newcommand\IsSort[1]{\IMode{#1}\ \textit{sort}}
\newcommand\Leq[2]{\IMode{#1}\leq\IMode{#2}}

\newcommand\Symbols{\mathbb{I}}
\newcommand\Finsets{\mathbb{F}}
\newcommand\Sorts{\mathcal{S}}
\newcommand\Valences{\mathcal{V}}
\newcommand\Arities{\mathcal{A}}
\newcommand\Operators{\mathcal{O}}

\newcommand\ParamsOf{\pi_\mathfrak{p}}
\newcommand\ArityOf{\pi_\mathfrak{a}}
\newcommand\Sets{\mathbf{Set}}

\newcommand\IsOperator[3]{\IMode{#1}\Vdash \IMode{#2} : \OMode{#3}}

\newcommand\IsMetaCtx[1]{\IMode{#1}\ \textit{mctx}}
\newcommand\IsVarCtx[1]{\IMode{#1}\ \textit{vctx}}
\newcommand\IsSymCtx[1]{\IMode{#1}\ \textit{sctx}}

\newcommand\Lookup[3]{\IMode{#1}\ni\IMode{#2}:\OMode{#3}}

\newcommand\IsAbt[5]{%
  \IMode{#1}\triangleright%
  \IMode{#2}\parallel%
  \IMode{#3}\vdash%
  \IMode{#4}:\OMode{#5}%
}

\newcommand\IsBTm[5]{%
  \IMode{#1}\triangleright%
  \IMode{#2}\parallel%
  \IMode{#3}\vdash%
  \IMode{#4}:\IMode{#5}%
}

\newcommand\MV[1]{\textsc{#1}}
\newcommand\MApp[3]{#1\left\{#2\right\}\left(#3\right)}
\newcommand\App[2]{#1\left(#2\right)}
\newcommand\Dom[1]{\left\vert #1\right\vert}
\newcommand\Fresh[2]{\IMode{#1}\notin\IMode{\Dom{#2}}}

\newcommand\SortExp{\mathsf{exp}}

\begin{document}

\title{Abstract Binding Trees}
\date{}
\author{Jonathan Sterling}
\maketitle

\section{Preliminaries}

Fix a set $\Sorts$ of \emph{sorts}. We will say $\IsSort{s}$ when
$\Member{s}{S}$. A valence $\MkValence{\vec{p}}{\vec{q}}{s}$ specifies an
expression of sort $s$ which binds symbols in $\vec{p}$ and variables in
$\vec{q}$.

\[
  \infer{
    \IsValence{\MkValence{p_0,\dots,p_m}{q_0,\dots,q_n}{s}}
  }{
    \IsSort{s} &
    \HypJ{\IsSort{p_i}}{\Leq{i}{m}} &
    \HypJ{\IsSort{q_i}}{\Leq{i}{n}} &
  }
\]

An arity $\MkArity{\vec{v}}{s}$ specifies an operator of sort $s$ with
arguments of valences $\vec{v}$. We will call the set of valences $\Valences$,
and the set of arities $\Arities$.

\[
  \infer{
    \IsArity{\MkArity{v_0,...,v_n}{s}}
  }{
    \IsSort{s} &
    \HypJ{\IsValence{v_i}}{\Leq{i}{n}}
  }
\]


Let $\Symbols$ be an infinite set of symbols; let $\Finsets$ be the free
cocartesian category over $\Symbols$. Then, fix a covariant presheaf of operators
$\Operators:\Finsets\times\Arities\to\Sets$ such that the arrows in $\Finsets$
lift to renamings of operators' parameters; via the Grothendieck construction,
we can also consider the set $\int\Operators$ of operators
$(U,\varrho,\vartheta)$ for $\Member{\vartheta}{\Operators(U,\varrho)}$.

\[
  \infer{
    \IsOperator{U}{\vartheta}{\varrho}
  }{
    \Member{\vartheta}{\Operators(U,\varrho)}
  }
\]

The judgment $\IsOperator{U}{\vartheta}{\varrho}$ enjoys the structural
properties of weakening and exchange via the functoriality of
$\Operators$.

\paragraph{Examples}

Operators are defined by specifying the fibres of $\int\Operators$ in which
they reside. For instance, consider the lambda calculus with a single sort,
$\SortExp$; we give its signature by asserting the following about its
operators:
\begin{gather*}
  \IsOperator{U}{\lambda}{
    \MkArity{\MkValence{\cdot}{\SortExp}{\SortExp}}{\SortExp})
  }\\
  \IsOperator{U}{\mathsf{ap}}{
    \MkArity{
      \MkValence{\cdot}{\cdot}{\SortExp},
      \MkValence{\cdot}{\cdot}{\SortExp}
    }{\SortExp}
  }
\end{gather*}

So far, we have made no use of symbols and parameters; however, consider the
extension of the calculus with assignables (references):

\begin{gather*}
  \IsOperator{U}{\mathsf{decl}}{
    \MkArity{
      \MkValence{\cdot}{\cdot}{\SortExp},
      \MkValence{\SortExp}{\cdot}{\SortExp}
    }{
      \SortExp
    }
  }\\
  \IsOperator{U,u}{\mathsf{get}[u]}{
    \MkArity{
      \cdot
    }{
      \SortExp
    }
  }\\
  \IsOperator{U,u}{\mathsf{set}[u]}{
    \MkArity{
      \MkValence{\cdot}{\cdot}{\SortExp}
    }{
      \SortExp
    }
  }
\end{gather*}

Declaring a new assignable consists in providing an initial value, and an
expression with a free symbol (which shall represent the assignable in scope).
Weakening can be seen as inducing a ``degeneracy map'' on operators, whereas a
renaming $u\mapsto v$ will take $\mathsf{get}[u]$ to $\mathsf{get}[v]$.

\section{Contexts}

In general, we will have three kinds of context: metavariable contexts,
variable contexts, and symbol (parameter) contexts. A metavariable context
$\Omega$ consists of bindings of valences to metavariables; a variable context
$\Gamma$ is a collection of bindings of sorts to variables, and a parameter
context $\Upsilon$ is a collection of bindings of sorts to symbols.

\begin{gather*}
  \infer{
    \IsMetaCtx{\cdot}
  }{
  }\qquad
  \infer{
    \IsMetaCtx{\Omega,\MV{m}:v}
  }{
    \IsMetaCtx{\Omega} &
    \IsValence{v} &
    \Fresh{\MV{m}}{\Omega}
  }
\end{gather*}

\begin{gather*}
  \infer{
    \IsVarCtx{\cdot}
  }{
  }\qquad
  \infer{
    \IsVarCtx{\Gamma,x:s}
  }{
    \IsVarCtx{\Gamma} &
    \IsSort{s} &
    \Fresh{x}{\Gamma}
  }
\end{gather*}

\begin{gather*}
  \infer{
    \IsSymCtx{\cdot}
  }{
  }\qquad
  \infer{
    \IsSymCtx{\Upsilon,u:s}
  }{
    \IsVarCtx{\Upsilon} &
    \IsSort{s} &
    \Fresh{u}{\Upsilon}
  }
\end{gather*}

\section{Abstract Binding Trees}

Let the judgment $\IsAbt{\Omega}{\Upsilon}{\Gamma}{M}{s}$ presuppose
$\IsMetaCtx{\Omega}$, $\IsSymCtx\Upsilon$, $\IsVarCtx\Gamma$ and $\IsSort{s}$,
meaning that $M$ is an abstract binding tree of sort $s$, with metavariables in
$\Omega$, parameters in $\Upsilon$, and variables in $\Gamma$. Let the judgment
$\IsBTm{\Omega}{\Upsilon}{\Gamma}{E}{v}$ presuppose $\IsValence{v}$.  Then, the
syntax of abstract binding trees is inductively defined in four rules:

\begin{gather*}
  \infer[\textit{var}]{
    \IsAbt{\Omega}{\Upsilon}{\Gamma}{x}{s}
  }{
    \Lookup{\Gamma}{x}{s}
  }\\[6pt]
  %
  \infer[\textit{mvar}]{
    \IsAbt{\Omega}{\Upsilon}{\Gamma}{
      \MApp{\MV{m}}{u_0,\dots,u_m}{M_0,\dots,M_n}
    }{s}
  }{
    \begin{array}{l}
      \Lookup{\Omega}{\MV{m}}{\MkValence{p_0,\dots,p_m}{q_0,\dots,q_n}{s}}\\
      \HypJ{\Lookup{\Upsilon}{u_i}{p_i}}{\Leq{i}{m}}\\
      \HypJ{
        \IsAbt{\Omega}{\Upsilon}{\Gamma}{M_i}{q_i}
      }{
        \Leq{i}{n}
      }
    \end{array}
  }\\[6pt]
  %
  \infer[\textit{app}]{
    \IsAbt{\Omega}{\Upsilon}{\Gamma}{
      \App{\vartheta}{E_0,\dots,E_n}
    }{s}
  }{
    \begin{array}{l}
      \IsOperator{\Dom{\Upsilon}}{\vartheta}{
        v_1,\dots,v_n
      }\\
      \HypJ{
        \IsBTm{\Omega}{\Upsilon}{\Gamma}{E_i}{q_i}
      }{
        \Leq{i}{n}
      }
    \end{array}
  }\\[6pt]
  \infer[\textit{abs}]{
    \IsBTm{\Omega}{\Upsilon}{\Gamma}{
      (\MkBTm{\vec{u}}{\vec{x}}{E})
    }{
      (\MkValence{\vec{p}}{\vec{q}}{s})
    }
  }{
    \IsAbt{\Omega}{\Upsilon, \vec{u}:\vec{p}}{\Gamma,\vec{x}:{\vec{q}}}{
      E
    }{
      s
    }
  }
\end{gather*}

\end{document}
