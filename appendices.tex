\section{$\SCtxSite$, Nominal Sets, and the Schanuel Topos}

Pitts defines a category $\mathbf{Nom}$ of nominal sets in \cite{pitts:2013}
which function equivalently to the sheaves that we considered above, in the
case of $\IMode{\Sorts}\cong\IMode{\mathbb{1}}$. Fix a countably infinite set
of atoms $\mathbb{A}$; then let $\Perm\mathbb{A}$ be the group of permutations
(i.e.\ autoequivalences) on $\mathbb{A}$.  Considered as a category,
$\Perm\mathbb{A}$ has a single object $\bullet$ with morphisms
$\Of{\pi}{\bullet\ra\bullet}$ for each $\Member{\pi}{\Dom{\Perm\mathbb{A}}}$.
Then, the category of nominal sets $\mathbf{Nom}$ is the subcategory of
$\Sets^{\Perm\mathbb{A}}$ containing just the presheaves which satisfy a
\emph{finite support} condition.

The category $\mathbf{Nom}$ is equivalent to the category of covariant sheaves
on $\SymSets$ under the atomic coverage, called the Schanuel Topos
$\Define{\mathbb{S}}{\Sheaves{\OpCat{\SymSets}}}$
\cite{pitts:2013,fiore-staton:2006}; equivalently, the Schanuel topos is the
subcategory of $\Sets^\SymSets$ containing only pullbacks-preserving functors.

In addition to the unisorted nominal sets framework, Pitts also briefly
discusses a multi-sorted version of the apparatus in which the countably
infinite set of atoms $\mathbb{A}$ is equipped with a sort assignment
$\Of{\mathfrak{s}}{\mathbb{A}\to\Sorts}$ such that all the fibers of
$\mathfrak{s}$ are countably infinite; then, he defines for any such assignment
the category $\mathbf{Nom}_\mathfrak{s}$ of $\mathfrak{s}$-sorted nominal sets
as a subcategory of the category of presheaves on the group of
$\mathfrak{s}$-respecting permutations on $\mathbb{A}$,
$\Perm_\mathfrak{s}\mathbb{A}$.

In the same way as the Schanuel topos is equivalent to the unisorted nominal
sets, we expect to find an equivalence between $\Sheaves{\SCtxSite}$ and the
limit $\int_{\mathfrak{s}}\mathbf{Nom}_\mathfrak{s}$. At the very least, the
sheaf topos $\Sheaves{\SCtxSite}$ may serve as a multi-sorted generalization of
the Schanuel topos, for which we may carry over certain useful results,
including its characterization as the subcategory of $\Sets^\SCtx$ which
contains just the pullbacks-preserving functors.


\begin{thm}
  \label{thm:pullback-preservation}
  A presheaf $\Of{X}{\Sets^{\SCtx}}$ is a sheaf on $\SCtxSite$ just when it
  preserves pullbacks.
\end{thm}

\begin{proof}

  The proof is essentially the same as that of the analogous lemma for the
  Schanuel topos as presented in \cite[A.2.1.11.h]{johnstone:2002}, but we will
  give a slightly more detailed version here. The presheaf $X$ is a sheaf on the
  atomic site $\SCtxSite$ just when, for any renaming
  $\Of{\varrho}{C\inj D}$ and any $\Member{n}{X(D)}$ if
  $\IsEq{n\cdot \varrho_0}{n\cdot \varrho_1}$ for all diagrams
  \[
    \begin{tikzcd}[sep=large]
      \IMode{C} \arrow[r, hook, "\IMode{\varrho}"] &
      \IMode{D}
        \arrow[r, hook, shift left, "\IMode{\varrho_0}"]
        \arrow[r, hook, swap, shift right, "\IMode{\varrho_1}"]
      &
      \IMode{E}
    \end{tikzcd}
  \]
  such that $\IsEq{\varrho_0\circ\varrho}{\varrho_1\circ\varrho}$, then there
  exists a unique $\Member{m}{X(C)}$ such that $\IsEq{m\cdot\varrho}{n}$. In other
  words, the arrow $X(\varrho)$ is an equalizer, as in the following:
  \[
    \begin{tikzcd}[sep=large]
      \IMode{X(C)} \arrow[r, "\IMode{X(\varrho)}"] &
      \IMode{X(D)}
        \arrow[r, shift left, "\IMode{X(\varrho_0)}"]
        \arrow[r, swap, shift right, "\IMode{X(\varrho_1)}"]&
      \IMode{X(E)}
      \\
      \IMode{\mathbb{1}}
        \arrow[u, densely dotted, "\OMode{m}"]
        \arrow[ur, "\IMode{n}"]
    \end{tikzcd}
  \]

  Now, because $\Sets$ is a regular category, $X(\varrho)$ is an equalizer
  precisely when it is a monomorphism. Therefore, if we have assumed that $X$
  preserves pullbacks, in order to demonstrate that $X$ is a sheaf, it suffices
  to show that $X(\varrho)$ is monic. If $X$ preserves pullbacks, then it also
  preserves monomorphisms, because in any category with pullbacks, $\Of{f}{A\ra
  B}$ is monic just when the square
  \[
    \begin{tikzcd}[sep=large]
      \IMode{A}
        \arrow[r, "\IMode{\ArrId{A}}"]
        \arrow[d, "\IMode{\ArrId{A}}"] &
      \IMode{A} \arrow[d, "\IMode{f}"]\\
      \IMode{A} \arrow[r, "\IMode{f}"] &
      \IMode{B}
    \end{tikzcd}
  \]
  is a pullback \cite[p.\ 16]{mac-lane-moerdijk:1992}. Because $X$ preserves
  monomorphisms, and all arrows in $\SCtx$ are monic, then $X(\varrho)$ is
  monic. Hence, $X$ is a sheaf.

  Next, we must show that if $X$ is a sheaf, then it preserves pullbacks; we
  will loosely track the proof of Proposition~2.6.15 in \cite{biering:2004}.
  Fix the following diagram such that it is a pullback square in $\SCtx$:
  \[
    \begin{tikzcd}[sep=large]
      \IMode{C}
        \arrow[r, "\IMode{f_j}"]
        \arrow[d, "\IMode{f_i}"] &
      \IMode{C_j} \arrow[d, "\IMode{s_j}"]\\
      \IMode{C_i} \arrow[r, "\IMode{s_i}"] &
      \IMode{D}
    \end{tikzcd}
  \]

  Then it suffices to show that the following diagram is also pullback square
  in $\Sets$:
  \[
    \begin{tikzcd}[sep=large]
      \IMode{X(C)}
        \arrow[r, "\IMode{X(f_j)}"]
        \arrow[d, "\IMode{X(f_i)}"] &
      \IMode{X(C_j)} \arrow[d, "\IMode{X(s_j)}"]\\
      \IMode{X(C_i)} \arrow[r, "\IMode{X(s_i)}"] &
      \IMode{X(D)}
    \end{tikzcd}
  \]

  We have to show that $X(C)$ is, up to equivalence, the pullback of $X(s_i)$
  along $X(s_j)$ in $\Sets$, namely:
  \[
    \IMode{X(C_i)\times_{X(D)}X(C_j)}
      \cong
    \OMode{
      \MkSet{
        (m_i, m_j)\in X(C_i)\times X(C_j)
          \mid
        m_i \cdot s_i = m_j \cdot s_j
      }
    }
  \]
  The first direction of the equivalence is trivial: for each
  $\Member{m}{X(C)}$, we can easily exhibit $\Member{(m\cdot f_i,m\cdot
  f_j)}{X(C_i)\times_{X(D)}X(C_j)}$.

  For the other direction, fix $\Member{(m_i,
  m_j)}{X(C_i)\times_{X(D)}X(C_j)}$.  Consider the singleton cover on $C$ that
  contains $\Define{h}{s_i\circ f_i}=\OMode{s_j\circ f_j}$; a matching family
  $\phi$ for the cover $\MkSet{h}$ is a map that takes $h$ to an object in
  $X(D)$, and an amalgamation for $\phi$ consists in an object
  $\Member{m}{X(C)}$ such that $\IsEq{\phi(h)}{m\cdot h}$. Now recall that $X$
  is a sheaf on $\SCtxSite$ if and only if every matching family has a unique
  amalgamation; for the matching family $\Define{\phi(h)}{m_i\cdot
  s_i}=\OMode{m_j\cdot s_j}$, then, we must have a unique amalgamation
  $\Member{m}{X(C)}$ since $X$ is a sheaf. Therefore, $X$ preserves pullbacks.

\end{proof}

